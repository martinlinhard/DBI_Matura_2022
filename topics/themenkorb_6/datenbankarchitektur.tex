\section{Datenbankarchitektur}

\subsection{Allgemeines}
\begin{itemize}
    \item Instanz $\implies$ Der Datenbankprozess + sämtliche Hintergrundprozesse, befindet sich im  Hauptspeicher
    \item Datenbank $\implies$ Besteht aus den Datenbankdateien
    \item SGA (System Global Area) $\implies$ Wird von allen Prozessen bzw. von allen Usern geteilt 
    \item PGA (Program Global Area) $\implies$ Pro Prozess
\end{itemize}

\subsection{System Global Area}
\begin{itemize}
    \item Besteht aus
    \begin{itemize}
        \item Database Buffer Cache
        \item Shared Pool
        \item Redo Log Buffer
        \item Data Dictionary Cache
    \end{itemize}
\end{itemize}

\subsubsection{Database Buffer Cache}
\begin{itemize}
    \item Speichert zuletzt verwendete Daten, um einen schnellen Zugriff gewährleisten zu können
    \item Befindet sich im RAM
    \item Besteht aus zwei Listen:
    \begin{itemize}
        \item Write-List: Enthält Daten, die modifiziert, aber noch nicht geschrieben wurden
        \item LRU-List: Enthält Adressen von zuvor modifizierten / freien Daten, welche lange nicht benutzt wurden und damit wieder zum Beschreiben verfügbar sind
    \end{itemize}S
    \item Drei Arten von Blöcken:
    \begin{itemize}
        \item Frei (schwarz)
        \item Belegt (weiß)
        \item Undo (für Rollback) (grau)
    \end{itemize}
\end{itemize}

\subsubsection{Shared Pool}
\begin{itemize}
    \item Enthält wichtige Daten von bereits ausgeführten SQL- und PL/SQL Anweisungen + das Data Dictionary
    \begin{itemize}
        \item \textbf{Data Dictionary} $\implies$ Metadaten zu Tabellen, Usern\dots
    \end{itemize}
    \item Dient dazu, um bei gleichen Anfragen schnell Ergebnisse liefern zu können
    \item Funktioniert nach dem \textbf{LRU} Prinzip
\end{itemize}

\subsubsection{Redo-Log Buffer}
\begin{itemize}
    \item Protokolliert sämtliche Änderungen im Database Buffer Cache $\implies$ Werden für die Wiederholung von Statements (Forward Recovery) benötigt
    \item Geschriebene Daten beschränken sich auf die wesentlichsten Informationen: Delete $\implies$ PK + Tabellenname\dots
    \item Hat eine fixe Größe (Ringbuffer) + ist dreigeteilt: Wenn das erste Drittel voll ist, beginnt der Logwriter (LGWR) das erste Drittel asynchron in die Online Redo-Log Dateien wegzuschreiben
    \begin{itemize}
        \item Wenn der Buffer voll ist + das erste Drittel noch nicht vollständig weggeschrieben wurde, muss gewartet werden!
    \end{itemize}
\end{itemize}

\subsubsection{Data Dictionary Cache}
\begin{itemize}
    \item Prüft vor Ausführung von Statements, ob die Spaltennamen existieren
\end{itemize}

\subsection{Prozesse}
\subsubsection{Database Writer (DBWR)}
\begin{itemize}
    \item Schreibt Daten aus dem Database Buffer Cache in die Daten-Dateien (standardmäßig alle 3 Sekunden, asynchron) (auch Tablespace genannt)
    \item Wählt die Daten aus, die am längsten nicht mehr verändert wurden (LRU)
    \item Schreibt, wenn:
    \begin{itemize}
        \item Database Buffer Cache erreicht gewisse Größe
        \item DBWR-Checkpoint wird erreicht
        \item ein Time-Out auftritt
        \item kein freier Buffer zur Verfügung steht
        \item eine Liste zum Abarbeiten vom LGWR kommt
    \end{itemize}
\end{itemize}

\subsubsection{Log Writer (LGWR)}
\begin{itemize}
    \item Schreibt Änderungen vom Redo Log Buffer in Redo Log Dateien $\implies$ Es kommen auch Undo-Informationen in die Redo-Logs!
    \item Schreibt sequentiell und wahlweise synchron / asynchron. Synchron ist kein Problem, da:
    \begin{itemize}
        \item Teile wurden bereits asynchron weggeschrieben
        \item Daten sind komprimiert
        \item LGWR schreibt schneller als DBWR
        \item Zusammenhängender Speicherplatz $\implies$ schreibt sequentiell
    \end{itemize}
    \item Commit $\implies$ LGWR schreibt gesamten Redo Log Buffer synchron; Commit ist erst fertig, wenn Redo Log Buffer leer ist
\end{itemize}

\paragraph{Log File Switch}
\begin{itemize}
    \item Eine Redo-Log Datei ist voll $\implies$ LGWR wechselt zu anderer Datei
    \item Zwei Modi:
    \begin{itemize}
        \item \textbf{archivelog} $\implies$ \textbf{ARCH} Prozess kopiert die vollen Online Redo Logs in die Offline Redo Logs
        \item \textbf{noarchivelog} $\implies$ Alte Dateien werden einfach überschrieben, falls der LGWR in seiner Rotation wieder bei einem vollen Log-File ankommt
        \item Änderung des Modus $\implies$ DBWR muss Blöcke so wegschreiben, dass die zu überschreibenden Redo-Logs nur Informationen enthalten, die bereits auf die Daten-Dateien übertragen wurden
    \end{itemize}
\end{itemize}

\subsubsection{Check Point (CKPT)}
\begin{itemize}
    \item Am Ende eines Checkpoints wird Header der Data- + Control-Files geupdated $\implies$ Datenbank kann so herausfinden, auf welchem Stand sich die Dateien befinden
\end{itemize}

\subsubsection{Archiver (ARCH)}
\begin{itemize}
    \item Archiviert automatisch die Online Redo-Log- Dateien in die Offline-Redo Logs
\end{itemize}

\subsubsection{System Monitor (SMON)}
\begin{itemize}
    \item Wird automatisch gestartet, wenn neue Datenbank-Instanz gestartet wird
    \item Schreibt Redo Logs von Änderungen, die noch nicht auf die Festplatte geschrieben wurden, damit bei Absturz die Instanz wiederhergestellt werden kann
    \item Führt automatischen Rollback aus, wenn Transaktionen nicht committed wurden
    \item Außerdem $\implies$ Alle 5 Minuten werden Wartungen durchgeführt, die Speicherplatz / temporäre Segmente freigeben
\end{itemize}

\subsubsection{Process Monitor (PMON)}
\begin{itemize}
    \item Zuständig, wenn ein User-Prozess versagt / abgebrochen wird $\implies$ reinigt Cache + gibt vom Prozess benutzte Ressourcen frei
\end{itemize}

\subsubsection{Recoverer (RECO)}
\begin{itemize}
    \item Zuständig, um Fehler, welche in ausgelagerten Transaktionen auftreten, zu beheben
\end{itemize}

\subsubsection{Lock Prozess (LCKn)}
\begin{itemize}
    \item Zuständig für das Sperren von Ressourcen zwischen verschiedener Instanzen in einem Server
\end{itemize}

\subsubsection{Dedicated Server Prozess}
\begin{itemize}
    \item Fertigt genau einen User ab
    \item Sinnvoll, wenn dieser eine User besonder viel macht
    \item Bei SELECT werden Daten direkt aus Database Buffer Cache / aus den Daten-Dateien geholt
\end{itemize}

\subsubsection{Shared Server-Prozess}
\begin{itemize}
    \item Kann mehrere User-Prozesse abfertigen
\end{itemize}

\subsubsection{Dispatcher}
\begin{itemize}
    \item Sorgt für Zuteilung zwischen User und Server
\end{itemize}

\subsubsection{Listener}
\begin{itemize}
    \item Verbindet Client zu Datenbank-Server
\end{itemize}

\subsection{Dateien}
\subsubsection{Daten-Dateien}
\begin{itemize}
    \item Enthalten Benutzer- und Systemdaten der Datenbank
\end{itemize}
\subsubsection{Control-Dateien}
\begin{itemize}
    \item Enthalten eine Beschreibung der Datenbankstruktur + ermöglichen Überprüfung und Sicherstellung der Integrität
\end{itemize}
\subsubsection{Redolog-Dateien}
\begin{itemize}
    \item Online / Offline
    \item Enthalten Befehle, welcher der User eingegeben hat, in einer komprimierten Form
\end{itemize}

\subsection{Fehlerbehandlung}
\subsubsection{Lokaler Fehler}
\begin{itemize}
    \item Rollback wird mithilfe der Undo-Blöcke im Database Buffer Cache durchgeführt
\end{itemize}

\subsubsection{Rollback}
\begin{itemize}
    \item Stromausfall $\implies$ SMON stellt aus Redo-Log Dateien wieder konsistenten Zustand her
\end{itemize}