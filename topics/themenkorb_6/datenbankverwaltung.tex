\section{Datenbankverwaltung}
\subsection{Allgemeines - Import}
\begin{itemize}
    \item \dots wenn Daten aus
    \begin{itemize}
        \item .csv
        \item .txt
        \item Excel-Files
        \item einem Datenbankbackup
    \end{itemize}
    \item importiert werden
    \item Herausforderung: Daten sind oft unnormalisiert, enthalte NULL-Werte\dots
\end{itemize}

\subsection{SQL Loader}
\begin{itemize}
    \item Wird dazu verwendet, um Daten in eine Oracle-DB zu importieren
    \item Verwendet Control-Dateien, die den Aufbau der Ausgangsdateien beschreiben
    \begin{itemize}
        \item Können auch vom SQL-Developer erstellt werden!
    \end{itemize}
    \item Beim Import werden sowohl erfolgreiche als auch gescheiterte Zeilen geloggt, welche dann später analysiert werden können
\end{itemize}

\subsubsection{Control-Dateien}
\begin{itemize}
    \item Beschreibt den Aufbau der zu importierenden Daten
    \item Es können Felder + Datentyp definiert werden, das Characterset festgelegt werden\dots
\end{itemize}

\subsection{Weitere Tools}
\begin{itemize}
    \item Oracle Import-Utility
    \begin{itemize}
        \item Es können Daten aus einer Exportdatei (dump) importiert werden $\implies$ Müssen zuvor mit der Export-Utility exportiert worden sein
    \end{itemize}
    \item Data Pump
    \begin{itemize}
        \item Neuer, schneller und flexibler als die Import-Utility; es kann eine PL/SQL API verwendet werden!
    \end{itemize}
\end{itemize}