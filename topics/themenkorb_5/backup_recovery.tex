\section{Backup \& Recovery}

\begin{itemize}
    \item Backup $\implies$ Kopie der Daten in einer Datenbank, um sie später wiederherzustellen
    \item Recovery $\implies$ Das Wiederherstellen der Daten selbst (im Fehlerfall)
\end{itemize}

\subsection{Backup}

\subsubsection{Arten von Backups}
Die Einteilung kann nach Menge der Daten / Häufigkeit der Backups und nach dem Zustand des Systems zum Backupzeitpunkt eingeteilt werden.
\paragraph{Menge der Daten, Häufigkeit der Backups}
\begin{itemize}
    \item Full Backup $\implies$ Es werden die gesamten Daten gesichert
    \begin{itemize}
        \item \textbf{Vorteil}: Restore-Zeit ist gering, hohe Redundanz $\implies$ sehr sicher
        \item \textbf{Nachteil}: Viel Speicherplatz wird benötigt, Anfertigung des Backups braucht viel Zeit
    \end{itemize}
    \item Partial Backup $\implies$ Es werden nur die Daten gesichert, die sich geändert haben
    \begin{itemize}
        \item \textbf{Vorteil}: Weniger Speicherplatz wird benötigt, Anfertigung des Backups braucht nicht so viel Zeit
        \item \textbf{Nachteil}: Hohe Restore-Zeit, Daten sind nur bedingt redundant $\implies$ geringere Sicherheit
        \item Unterarten
        \begin{itemize}
            \item Differential Backup $\implies$ Es werden nur Daten gespeichert, die sich seit dem letzten \textbf{Full Backup} geändert haben
            \begin{itemize}
                \item \textbf{Vorteil}: Restore-Zeit ist gering, sicherer da eine gewisse Rendundanz gegeben ist
                \item \textbf{Nachteil}: Viel Speicherplatz wird benötigt, Anfertigung des Backups braucht viel Zeit
            \end{itemize}
            \item Incremental Backup $\implies$ Es werden nur Daten gespeichert, die sich seit dem letzten \textbf{Partial Backup} geändert haben
            \begin{itemize}
                \item \textbf{Vorteil}: Weniger Speicherplatz wird benötigt, Anfertigung des Backups braucht nicht so viel Zeit
                \item \textbf{Nachteil}: Restore dauert länger, es gibt wenige Redundanzen $\implies$ Nicht sehr sicher
            \end{itemize}
        \end{itemize}
    \end{itemize}
\end{itemize}

\paragraph{Zustand des Systems zum Backupzeitpunkt}
\begin{itemize}
    \item Online (Hot) Backup
    \begin{itemize}
        \item Wird während dem laufenden Betrieb ausgeführt
        \item Achtung: Während der Erstellung des Backups können Änderungen geschehen $\implies$ Es müssen vor Beginn des Backups die Änderungen mitprotokolliert werden
        \begin{itemize}
            \item Um konsistenten Zustand einzuspielen $\implies$ Redo Logs müssen ausgeführt werden
        \end{itemize}
    \end{itemize}
    \item Offline (Cold) Backup
    \begin{itemize}
        \item Wird ausgeführt, wenn die Datenbank offline ist
    \end{itemize}
\end{itemize}

\subsection{Recovery}
\subsubsection{Arten von Fehlersituationen}

\paragraph{Transaktionsfehler (Lokaler Fehler)}
\begin{itemize}
    \item Transaktion wurde nicht ordentlich beendet; Daten sind nun in inkosistentem Zustand
    \item Auslöser
    \begin{itemize}
        \item Runtime-Fehler
        \item Deadlock
        \item Time-Out
        \item Manuelles Rollback\dots
    \end{itemize}
    \item Maßnahmen
    \begin{itemize}
        \item Alle Änderungen bis hin zum Abbruch müssen zurückgenommen werden (Rollback / Transaction Recovery) $\implies$ Backward Recovery 
    \end{itemize}
\end{itemize}

\paragraph{Systemfehler (Soft Crash)}
\begin{itemize}
    \item Mehrere Transaktionen konnten nicht ordnungsgemäß beendet werden
    \item Auslöser
    \begin{itemize}
        \item Stromausfall
        \item Fehler im Betriebssystem
    \end{itemize}
    \item Maßnahmen
    \begin{itemize}
        \item Alle Änderungen der Transaktionen, die beim Absturz in Progress waren, müssen zurückgenommen werden (Crash Recovery) $\implies$ Backward Recovery
    \end{itemize}
\end{itemize}

\paragraph{Mediumfehler (Hard Crash)}
\begin{itemize}
    \item Daten sind physikalisch zerstört / nicht mehr lesbar
    \item Auslöser
    \begin{itemize}
        \item Irrtümliches Löschen von Daten
        \item Fehler in der Dateiverwaltung des Betriebssystems
        \item Fehler im Disk Controller
    \end{itemize}
    \item Maßnahmen
    \begin{itemize}
        \item Sicherungsstand wird eingespielt, Änderungen seit Sicherung müssen nachvollzogen werden (Media Recovery, Disaster Recovery, Crash Recovery) $\implies$ Forward Recovery
    \end{itemize}
\end{itemize}

\subsubsection{Techniken für Recovery}
\paragraph{Backward Recovery}
\begin{itemize}
    \item Vor sämtlichen Änderungen innerhalb der Datenbank wird eine Kopie von den alten Werten (= Before Image) erstellt \& in Undo-Log-Dateien gesichert $\implies$ Nicht abgeschlossene Transaktionen können so rückgängig gemacht werden!
    \item Undo Log enthält unter Anderem:
    \begin{itemize}
        \item Identifikation der Transaktion
        \item Art der Operation (Insert\dots)
        \item Before Image
    \end{itemize}
    \item Logs werden meist zu Recovery-Zwecken fortlaufend geführt
    \item Transaction Recovery
    \begin{itemize}
        \item Der Undo-Log wird bis zum Beginn der Transaktion gelesen
    \end{itemize}
    \item Crash Recovery
    \begin{itemize}
        \item Der Undo-Log wird bis zum Beginn gelesen, um alle Before-Images von nicht-beendeten Transaktion zu finden
        \item Undo-Log wird in Checkpoints unterteilt (=zu diesem Zeitpunkt aktive Transaktionen werden gespeichert)
        \item Undo-Log wird bis zum jüngsten Checkpoint gelesen $\implies$ Alle Transaktion dieses Checkpoints, welche keine Endmarke haben, werden rückgängig gemacht!
    \end{itemize}
\end{itemize}

\paragraph{Forward Recovery}
\begin{itemize}
    \item 2 Stategien
    \item Logging
    \begin{itemize}
        \item Sämtliche Änderungen werden nach Ende der Transaktion als After Images in Redo-Log Dateien gespeichert
    \end{itemize}
    \item Gespiegelte Platten
    \begin{itemize}
        \item RAID
    \end{itemize}
\end{itemize}

\section{Data Control Language}
\begin{itemize}
    \item Wird verwendet, um gewissen Usern bestimmte Berechtigungen zu geben
\end{itemize}