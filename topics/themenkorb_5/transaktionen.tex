\section{Transaktionen}
\begin{displayquote}
    "Eine Folge von Datenbankanweisungen, welche entweder ganz oder garnicht ausgeführt wird."
\end{displayquote}
\subsection{Allgemeines}
\subsubsection{ACID Prinzip}
\begin{itemize}
    \item Atomicity $\implies$ Transaktion ist die kleinste Arbeitseinheit, sie wird entweder ganz oder garnicht ausgeführt
    \item Consistency $\implies$ Die Datenbank ist zu Beginn und Ende jeder Transaktion konsistent
    \item Isolation $\implies$ Änderungen innerhalb einer Transaktion sind nur für diese sichtbar!
    \item Durability $\implies$ Nach Beendigung einer Transaktion (successful commit) sind die Daten dauerhaft, auch im Fehlerfall, gespeichert.
\end{itemize}
\subsubsection{Commit und Rollback}
\begin{itemize}
    \item Commit $\implies$ Transaktion wird beendet, Änderungen werden dauerhaft gespeichert!
    \begin{itemize}
        \item Änderungen sind nun für alle sichtbar!
    \end{itemize}
    \item Rollback $\implies$ Änderungen seit dem letzten Commit werden verworfen!
    \item AutoCommit $\implies$ Nach jeder Anweisung wird ein Commit ausgeführt, sofern die Anweisung erfolgreich ausgeführt wurde
    \begin{itemize}
        \item Nicht erfolgreich $\implies$ Automatisches Rollback!
        \item Modus wird deaktiviert, wenn explizite / implizite Transaktion gestartet wird! 
    \end{itemize}
\end{itemize}
\subsubsection{DDL Statements - Implicit Commit}
\begin{itemize}
    \item Achtung: Sämtliche DDL Statements (Create Table\dots) führen automatisch zu einem Commit!
    \begin{itemize}
        \item Zuvor ausgeführte Änderungen werden zuerst comitted, DDL-Statements dann in einer neuen Transaktion!
    \end{itemize}
\end{itemize}
\subsubsection{Länge von Transaktionen}
\begin{itemize}
    \item so kurz als möglich, da:
    \begin{itemize}
        \item Tabellen nicht so lang gesperrt bleiben müssen
        \item Weniger Statements im Fehlerfall wiederholt werden müssen
        \item Allgemein weniger Overhead ensteht!
    \end{itemize}
    \item so lang als notwendig, damit die Daten konsistent sind!
\end{itemize}

\section{Anomalien im Einbenutzerbetrieb}
\begin{itemize}
    \item Es kann beim Einfügen, Updaten und Löschen zu Problemen kommen, wenn die Daten nicht in die 3. Normalform gebracht wurden!
\end{itemize}